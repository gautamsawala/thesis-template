%!TEX root = ../Thesis.tex
\paragraph{}
Wearable devices able to collect high quality vital physiological signals will be commercially available and affordable in the future. While these wearable devices have several health benefits, they also pose a privacy concern.
\paragraph{}
In this thesis, we devised a physiological signal based short-movie recognition system. The physiological signals taken into consideration were Electrocardiogram (ECG) and Electrodermal activity (EDA). We review the literature of these two physiological signals to understand their significance for our study. We devised an experimental setup to collect the physiological signals for our study. The physiological data from 72 subjects were collected for the analysis.
\paragraph{}
The physiological data was filtered using widely accepted filtering algorithms. Time domain features were extracted for EDA signals and frequency domain features were extracted for ECG signals. A machine learning algorithm, the random forest was used to classify the ECG features. Two deep learning neural network models, base LSTM and CNN-LSTM were implemented to classify the EDA signals.
\paragraph{}
\todo{Replace the results}
Our results show a better than the probability of chance prediction accuracy of 39\% for random forest classification of 7 movies. The base LSTM show prediction accuracy of x\%, x\%, x\%, x\%, x\%, x\% for classification of 2, 3, 4, 5, 6, 7 movies respectively. The CNN-LSTM show prediction accuracy of 61\%, 41\%, 38\%, 38\%, x\%, 21\%  for classification of 2, 3, 4, 5, 6, 7 movies respectively. Thus, we conclude that there exists some correlation between physiological data and user's viewing activity. 