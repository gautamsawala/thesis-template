
\section{Conclusion}
The wearable devices which are capable of measuring vital physiological data like Electrocardiogram (ECG) and Electrodermal activity (EDA) are now closer to realization \cite{haghi_wearable_2017}. In this thesis, we devise an approach on how these vital physiological signals collected by wearable devices could be used to exploit the user's privacy. We reviewed the literature to understand the effect of external stimuli of ECG and EDA and their underlying principles.

\paragraph{} We did not find any readily available dataset to suit our requirements. The data used in the study was collected with the biosignal acquisition device, BITalino. We designed a custom software application to accurately tag the physiological data corresponding to the movie. We performed several tests to overcome challenges like time synchronization while acquiring data from multiple BITalino devices, determining the best placement for electrodes for collecting physiological signal acquisition, static noise removal, parallel data collection of multiple subjects, etc. Hence, we devised an experimental setup for data collection and were able to collect data from 72 subjects. The summary of the dataset can be found in the table \ref{tab:ecg_data_set} and table \ref{tab:eda_data_set}.

\paragraph{} The physiological data was then filtered using the techniques and algorithms widely used in physiological data research. The features were then extracted from the filtered data. To the best of our knowledge, there is no previous research done in regards to the correlation between the subject's viewing activity and their physiological signals. So we did not have any pointers to which features of physiological data are vital for determining the subject's viewing activity. However, there is ample of research done in the area of identifying the subject's emotions based on the physiological signals. We thus reviewed the literature in this area to determine and extract the significant features of physiological data.

\paragraph{} We implemented a popular machine learning algorithm, random forests, to analyze the features extracted in the frequency domain from the ECG and a deep learning algorithm based on neural networks, Long Term Short Memory (LSTM) and it's a variation to analyze the features extracted from EDA. We took inspiration from research done in other domains to establish our models and train the data. Our analysis showed an average prediction accuracy of 39\% for random forest model implemented on features extracted from ECG for 7 movies. The LSTM base model implemented on phasic EDA feature extracted from EDA showed an average prediction accuracy of 19\% and the CNN-LSTM model showed an average prediction accuracy of 24\% for six movies. We would like to note that even though the accuracy is not high, it is still better than the probability of the prediction by chance which is 16.6\% for 6 movies. Hence, we meet our objective and deduce that there exists some correlation between a user's viewing activity and their physiological data. Meanwhile, there are several improvements that can be done which we were unable to address due to the limited time of the thesis. We discuss these improvement in section \ref{sec:future_work}

\section{Future Work}
\label{sec:future_work}
While reviewing the feedback section of the questionnaires we received several requests asking about the research and the results of this study. Presuming that this study would pave the way and spark interest in others to instigate research in this domain, we pass on the baton with our suggestions on improving on the study and future work.

\paragraph{Electrode placement} We noticed a significant variance in the amplitude of different ECG signals. Since we extracted features for ECG in the frequency domain it did not matter to us. However, uniformity must be maintained in electrode placement for different subjects if the features are to be extracted in the time domain.

\paragraph{} During our test phase while devising the experimental setup, we noticed that EDA for some subjects exceeded the threshold. We overcome this by altering the position of electrodes on the finger of the subject. However, we still had to eliminate some of the samples as they exceeded the threshold in the middle of the session. Using a physiological data collection device with a higher threshold would eliminate this problem altogether.

\paragraph{Data Collection} The models can be trained much better with the availability of larger dataset with wider demographics of people. Secondly, longer movies can be taken into consideration for data collection. However, care must be taken with the selection of the movies as we observed in our study that the subjects got distracted or lost interested after fifteen to twenty minutes. We observed this as the variation in the EDA signals decreased as the subjects had lost interest in the movie. A theatrical ambiance may be best suited for longer movies to hold the subject's attention.

\paragraph{Scaling and Filtering} There are several methods available to filter the physiological data. Other methods than one used in our study can be experimented with to get better signal quality. Another important aspect is the scaling of the data. Other scaling methods can be utilized to normalize the physiological signals. Furthermore, the baseline of the subject can be taken into consideration while filtering and scaling.

\paragraph{Feature Selection} The features of EDA can be extracted in the frequency domain and features of the ECG can be extracted in the time domain. These features might have some significance if analyzed. Combining both EDA and ECG for analysis can further increase the accuracy. 

\paragraph{Analysis} The Long Term Short Memory (LSTM) neural network seems to be promising. A further tweak in the variables and fine-tuning the neural network might give a better result. DeepConvLSTM as introduced by \citeauthor{ordonez_deep_2016} can be utilized for this analysis. Further addition of dense and convolution layers may also help.

\paragraph{} In our proposed weight based classification algorithm, improvements can be made by using supervised learning methods. Manual weights can be assigned for the time at which the onset is expected based on the kind of scene in the movie. A larger dataset for such a method would also help reduce the bias in the weights and reduce false-positive classification.

\paragraph{Closing Remarks} We believe that the implications of our work will further the research in this area. The wearable devices in future will be able to collect high quality physiological signals. Thus, it will become all the more important to understand the privacy implication of these wearable devices. 
