%!TEX root = ../Thesis.tex
\paragraph{}
A significant amount of research conducted by Picard et al. at the Massachusetts Institute of Technology (MIT) Laboratory, showing that certain affective states may be recognized by using physiological data.\cite{picard_toward_2001}. Thus, implying that our emotions affect physiological state of our body.
\paragraph{}
J Kim et al. designed a physiological signal-based emotion recognition system. The user-independent takes in Electrocardiogram (ECG), Skin Temperature (ST) variation and Electrodermal Activity (EDA) and were able to achieve 78.4\% and 61.8\% for the recognition of three and four categories of emotions, respectively.\cite{kim_emotion_2004}
\paragraph{} 
Emotions in an individual are induced by external stimuli. There is fair amount research done on understanding emotions of an individual based on physiological data. However, there is significant lack of research to determine the stimuli causing the emotion based on the physiological data.
\paragraph{}
Our goal is to understand if we can determine the stimuli, in our case the short movies, based on the raw physiological data, which today’s wearable devices collect. Being able to determine the stimuli based on the physiological data, opens an entirely new avenue into privacy concern.